\documentclass[../../Report.tex]{subfiles}
\usepackage[italian]{babel}

\begin{document}
\chapter{Data Mining}
\section{Dataset}
In nostro possesso è stato dato un dataset in formato \emph{.csv} contenente 62.584 osservazioni e 13 colonne riguardanti il valore dei sensori di uno smartphone android e l'attività dell'utente.\\
Le etichette di questo dataset assumono 5 valori differenti:
\begin{itemize}
    \item \textbf{STILL}: l'utente non si muove.
    \item \textbf{WALKING}: l'utente cammina.
    \item \textbf{CAR}: l'utente è in auto.
    \item \textbf{BUS}: l'utente è in bus.
    \item \textbf{TRAIN}: l'utente è in treno. 
\end{itemize}
Le feature del dataset sono 12 e sono le seguenti:
\begin{itemize}
    \item \textbf{accelerometer\#mean}: la media delle osservazioni dell'accelerometro.
    \item \textbf{accelerometer\#min}: l'osservazione minima dell'accelerometro.
    \item \textbf{accelerometer\#max}: l'osservazione massima dell'accelerometro.
    \item \textbf{accelerometer\#std}: la deviazione standard delle osservazioni dell'accelerometro.
    \item \textbf{gyroscope\#mean}: la media delle osservazioni del giroscopio.
    \item \textbf{gyroscope\#min}:  l'osservazione minima del giroscopio.
    \item \textbf{gyroscope\#max}:  l'osservazione massima del giroscopio.
    \item \textbf{gyroscope\#std}: la deviazione standard delle osservazioni del giroscopio.
    \item \textbf{gyroscopeuncalibrated\#mean}: la media delle osservazioni del giroscopio non calibrato.
    \item \textbf{gyroscopeuncalibrated\#min}:  l'osservazione minima del giroscopio non calibrato.
    \item \textbf{gyroscopeuncalibrated\#max}: l'osservazione massima del giroscopio non calibrato.
    \item \textbf{gyroscopeuncalibrated\#std}: la deviazione standard delle osservazioni del giroscopio non calibrato.
\end{itemize}



\end{document}