\documentclass[../../Report.tex]{subfiles}
\usepackage[italian]{babel}

\begin{document}
    \chapter{Progettazione}
    Nella prima fase di questo progetto, abbiamo eseguito un'analisi approfondita dei requisiti specifici. Abbiamo identificato le seguenti esigenze per l'app mobile, il server e il frontend:
    \begin{itemize}
        \item \textbf{Applicazione mobile:} L'applicazione mobile deve essere in grado di riconoscere l'attività dell'utente, la variazione dell'attività di un utente (WALKING $\rightarrow$ DRIVING e viceversa), visualizzare una mappa con le varie aree poligonali di Bologna, chiedere al backend la disponibilità di parcheggi all'interno di una certa area poligonale, riconoscere la posizione dell'utente e inviare al server gli eventi di entrata e uscita di un parcheggio.
        \item \textbf{Server:} Il server deve essere in grado di ricevere e rispondere alle richieste di disponibilità di parcheggi da parte dell'app, ricevere le richieste di entrata e uscita di parcheggi da parte dell'app, inviare le aree poligonali di Bologna all'app e, in caso di zone prive di dati (ad esempio per mancanza di utilizzo dell'app da parte di utenti in quella zona) deve prevedere un meccanismo di interpolazione dei dati, per stimare il numero di parcheggi disponibili.
        \item \textbf{Frontend:} Il frontend deve essere in grado di visualizzare, per ogni zona della città, il numero totale di richieste di parcheggio provenienti da quella zona, utilizzare il clustering K-Means per visualizzare gli eventi di parcheggio e rappresentare gli eventi di parcheggio mediante una heatmap.
    \end{itemize}

    Per quanto riguarda la seconda fase, è stato richiesto di valutare l'accuratezza di tre classificatori di attività binari. In particolare si è lavorato su un dataset fornito per allenare il modello di Human Activity Recognition (HAR). Sono stati testati tre algoritmi di classificazione: Random Forest, KNN, Guassian Bayes. Inoltre, è stata valutata l'accuratezza del sistema di HAR della libreria nell'app mobile mediante un piccolo testbed (esperimenti con pochi dati reali) e confrontata con l'accuratezza degli algoritmi testati, tenendo presente che il confronto ha una valenza scientifica limitata in quanto si basa su dataset differenti. Infine, è stato integrato il Random Forest direttamente nell'app mobile prendendo i dati dai sensori e sono state sviluppate due feature aggiuntive a discrezione del gruppo.
    
\end{document}