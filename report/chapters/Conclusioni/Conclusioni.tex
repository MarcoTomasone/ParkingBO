\documentclass[../../Report.tex]{subfiles}
\usepackage[italian]{babel}

\begin{document}
\chapter{Conclusioni}
Lo scopo di questo progetto era quello di costruire un'applicazione di crowdsensing dei parcheggi per la città di Bologna. Il progetto prevedeva l'implementazione di un'applicazione mobile, un backend ed un frontend. Lo scopo principale del progetto era quello di sperimentare con delle tecniche location-aware e activity-aware andando a lavorare con la posizione dell'utente (per il riconoscimento della zona di parcheggio) e con le attività svolte dall'utente (per la gestione del riconoscimento di entrate e uscite dal parcheggio). Per rispettare queste proprietà oltre alle specifiche abbiamo deciso di estendere il progetto implementando un sistema di riconoscimento dell'utilizzo delle colonnine di ricarica elettrica e di implementare come seconda feature aggiuntiva l'algoritmo di clustering DBSCAN, oltre a quello K-Means. Il progetto è stato molto interessante e ci ha dato la possibilità di sperimentare nuove tecnologie e di approfondire le conoscenze acquisite durante il corso. 
\end{document}


