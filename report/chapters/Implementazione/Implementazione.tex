\documentclass[../../Report.tex]{subfiles}
\usepackage[italian]{babel}

\begin{document}
\chapter{Implementazione}
\label{chap:Implementazione}
In questa parte del report verranno descritte le tecnologie utilizzate per la realizzazione del sistema e le scelte fatte per la sua implementazione.\\

\section{App Mobile}
    Per offrire un'app mobile che possa essere utilizzata sia su sistema operativo Android che iOS abbiamo deciso di utilizzare il framework \emph{Flutter}, un framework open-source sviluppato di Google che si poggia su \emph{Dart}. Il progetto è stato però testato solo su android poichè nessun componente del gruppo possedeva un dispositivo con MacOS. Per la gestione della mappa abbiamo utilizzato il pacchetto \emph{flutter\_map} che permette di integrare comodamente delle mappe Leaflet all'interno delle propria app, mentre per quanto riguarda la libreria per fare Activity Recognition abbiamo utilizzato un pacchetto chiamato \emph{flutter\_Activity\_Recognition}, un wrapper alle librerie di sistema Android e iOs per fare riconoscimento dell'attività dell'utente. \\

\section{Backend}
    Per il backend abbiamo deciso di utilizzare \emph{Node.js}, un runtime Javascript che permette di creare applicazioni web veloci e scalabili, il Database invece, è stato creato utilizzando \emph{PostgreSQL} con estensione \emph{PostGIS} per la gestione dei dati spaziali.\\

\section{Frontend}
    Per  motivi di riusabilità del codice e dei componenti abbiamo deciso di utilizzare il framework React per il frontend, un framework \emph{Javascript} open-source sviluppato da Facebook che permette di creare componenti riutilizzabili. 
    Per la gestione delle mappe sia nel frontend abbiamo utilizzato la libreria \emph{Leaflet}, un framework Javascript open-source che permette di creare mappe interattive, dal quale abbiamo preso anche due pacchetti per fare Clustering e per visualizzare la Heatmap dei dati. Infine per la parte grafica è stato utilizzato il framework \emph{Material-UI}, un framework Javascript open-source che permette di creare componenti grafici in stile \emph{Material Design}.\\


\end{document}